\documentclass[a4paper,11pt]{article}
\usepackage[utf8]{inputenc}
\usepackage{textcomp}
\usepackage{lmodern}
\usepackage{listings}
\usepackage{graphicx}
\usepackage{listings}
\usepackage{color}
\usepackage{url}
\usepackage{verbatim}
\usepackage[top=3cm,bottom=3cm,left=3cm,right=3cm]{geometry}

\title{Master in Cyber-Security\\
	LINGI2146 --- Mobile and Embedded Computing: \\
	Publishing IoT Sensor Data through a MQTT publish/subscribe infrastructure}

\author{FONTAINE Romain, NYAKI Loïc, TIO NOGUERAS Gérard}



\begin{document}
\maketitle
\newpage
\tableofcontents

\newpage

\section{Introduction}
When using sensors to measure data, such as light, temperature or noise, it is common to use a IoT device: a small battery-powered piece of hardware, equipped with one or several sensors, that can communicate data wirelessly.\\

The way to propagate information is by communicating with neighbouring devices, which will forward the information towards a specific node of the network, called the \textit{root node}. To be able to communicate in this manner, each device must form a relationship with the other nodes and form what is called a Wireless Mesh Network (WMN).\\

Another part of this system is the MQTT infrastructure. MQTT (Message Queuing Telemetry Transport) is a publish/subscribe messaging protocol that allows publishers to publish data to a MQTT broker, while clients can subscribe to that same broker, to receive specific information from the publishers.\\

The first goal of this project is to implement a custom routing protocol, similar to RPL, that will allow the creation of a network of IoT devices. Information should be able to travel from any node of the network to the root node which acts as a gateway towards the outside world. \\

The second goal is to create an MQTT infrastructure, that allow clients to subscribe to a MQTT broker, and publishers to publish data to the broker.\\

The third and last goal is then to create a \textit{Gateway agent} that will serve as an interface between the IoT network and the MQTT infrastructure, in such a way that the IoT devices should be able to become MQTT publishers, and any client from outside the IoT network should be able to subscribe to the MQTT broker, and be periodically updated with data from the IoT network sensors.

\begin{comment}
First, we need implement a custom routing protocol for sensor networks, similar to RPL, on top of 6LoWPAN which is a special version of IPv6 for low-power devices. Then, we build a messaging infrastructure using MQTT, which is a publish/subscribe messaging protocol. Lastly, we interface the sensor network with the MQTT message infrastructure, by using a border router, whose job will be to translate the message travelling between the sensor network and the MQTT message infrastructure.
\\
Lastly, the sensor data 
\end{comment}
\section{Overview of the System}
The scope of this project covers three separate part (see figure \ref{fig:architecture1}):
\begin{itemize}
\item{An IoT sensor network, where devices organize themselves, and communicate with each other by using a custom routing protocol similar to RPL.}
\item{An MQTT publish/subscribe infrastructure composed of an MQTT broker and several MQTT clients that can either publish data or/and subscribe to topics to receive data.}
\item{A Rime-MQTT Gateway which interfaces the sensor network with the MQTT infrastructure in order for the IoT network and MQTT infrastructure to talk to each other in a transparent manner.}
\end{itemize}

\begin{figure}
  \includegraphics[width=\linewidth]{img/architecture-2.jpg}
  \caption{An overview of the system.}
  \label{fig:architecture1}
\end{figure}
\section{Network Structure and Routing Protocol}
One of the objectives of this project is the implementation of a routing protocol, similar to RPL but simplified where nodes can only have 1 parent.\\
This will create a tree based structure.\\
We have to prepare our protocols for a list of scenarios and make sure our implementation is capable to handle them.

\subsection{Joining a graph}
\label{sec: joining-graph}
When a node tries to join a graph it will broadcast a \textit{Discovery message}to inform sourrounding nodes of its intent. It then waits for a certain time for a response. The nodes receiving the broadcast and already having a parent will unicast back to the new node with a \textit{Parent Message} and their rank. The new node will add the different parents and after the timer is finished, go through the "Vador propositions" and choose the one with the smallest rank. It will sends a unicast \textit{Parent-ACK Message} to the parent, to confirm their child-parent relationship. The parent then adds the child to the list of its children.\\
If it doesn't receive any response, it loops on the discovery broadcast.\\
When getting a new parent, the child node also inherits the configuration(periodic or update mode and activ) of the parent and is therefore up-to-date with the rest of the tree.

\subsection{Leaving a Graph}
Nodes that are in a parent-child relationship maintain an awareness to each other. The child periodically pings its parent with an \textit{Information Message}. The parent then pongs back with \textit{Information-Acknowledgement Message}.\\
If no ACK is sent back, the child will try to ping its parent a total of three times with timers in between. After that delay, the parent is considered missing and the child starts looking for a new parent by restarting the discovery loop process. (see section \ref{sec: joining-graph}).

\subsection{Sending Sensor Data}
Periodically, each node uses a \textit{ROOT message} to forward data to the root node of the Tree which will then forward it to the Gateway where it will be translated into an MQTT message that will be sent to the MQTT broker. Once it reaches the MQTT broker, the message is forwarded to all the MQTT client that have subscribed to the specific topic relative to that message.
The forwarding of the sensor data to the root works because by forwarding the sensor data to the parent of any node it will reach at some point the root.

\subsection{Sending Configuration Requests}
We can send configuration requests through a MQTT publisher, that publishes on a "/config" topic, to which the Rime-MQTT Gateway is subscribed. The Gateway then translate that MQTT configuration message into a message that the sensor network can understand which then is sent to the root that will propage the new configuration by sending unicast messages to each nodes sons recursively.

\subsection{Summary of Message Types}
The following section describes all the message types that are used by the IoT devices in order to organize themselves and communicate.

\subsubsection{TAG-DISCOVERY: Discovery Message}
When a node doesn't belong to a network, it will broadcast request for potential parent nodes to manifest themselves. This is called a \textit{Discovery message}.

\subsubsection{TAG-VADOR: Parent Message}
A \textit{Parent Message} is sent by a member of the network after receiving a Discovery message. This message advertises the parent's address as well as its rank within the network.

\subsubsection{TAG-ACK-PARENT: Acknowledgement of Parent message}
After receiving one or more Parent Messages and chosing a parent, the new node will then send an ACK to its new parent. This will allow the parent to add him as a child for future configuration updates.

\subsubsection{TAG-INFO: Information Message}
An \textit{information Message} is a ping from a child to its parent to assess the state of their connection.

\subsubsection{TAG-ACK-INFO: Acknowledgment of Information message}
This is a pong back to the child's ping. This serves as a confirmation that the message has been properly received.


\subsubsection{TAG-ROOT: Messages destined to the root node}
Periodically, each node sends takes information from it's sensors and sends it to its parent. To do this, an \textit{Information message} is used.


\subsubsection{UPDATE-MODE: }
\subsubsection{PERIODIC-MODE: }
\subsubsection{START-SEND:}
\subsubsection{STOP-SEND:}


\subsection{Root Node}


\section{MQTT infrastructure}
MQTT stands for Message Queuing Telemetry Transport, and is a publish/subscribe messaging protocol. It requires a MQTT broker as well as MQTT clients that can publish information to the broker, as topics, or subscribe to the broker for a specific topic, in order to be updated every time there is new information on that topic.

\subsection{MQTT Broker}
The broker is the central element of MQTT. It is where each MQTT client either subscribe to a topic, or publishes information under a specific topic. In this instance, we used Mosquitto, an open-source MQTT broker that will be running on a regular laptop.


\subsection{MQTT Publishers}
In this project, we have two distinct MQTT publishers. The first MQTT publisher is the \textbf{Rime-MQTT Gateway}, which publishes topics on behalf of the sensor network nodes.\\

The second publisher is the \textbf{MQTT client} used for configuring the sensor network. We use this MQTT configuration client to send configuration commands to the sensor network, in place of using a direct command line interface on the Gateway machine.

\subsection{MQTT Subscribers}
The MQTT broker allows for clients to both publish data, or subscribe in order to be kept updated on data. In this project, we have two different MQTT subscribers : regular MQTT clients, and the Rime-MQTT Gateway that will subscribe to the configuration topic.

\subsection{Regular MQTT clients}
For demonstration purpose, one or several clients will run on our machine, and subscribe to specific topics, to receive information from the sensor network. Once subscribed to the appropriate topic, each subscriber will receive periodical information from the network.

\subsection{Rime-MQTT Gateway as a client}
In order to implement the possibility of modifying the IoT network configuration dynamically, we need the Rime-MQTT Gateway to subscribe to a "configuration" topic to the MQTT broker. A specific client will then, when necessary publish configuration commands that will be forwarded to the Gateway.


\section{Rime-MQTT Gateway}
The Rime-MQTT Gateway is a python program, that serves as an interface between the sensor network and the MQTT broker.\\

On one side, it receives data from the root node of the sensor network and translates them into MQTT publish messages that it sends to the MQTT broker. On the other side, it subscribes to the MQTT broker for configurations messages that it then translate into a format that the root of the sensor node understands (see figure \ref{fig:communication1}).\\


\subsection{Prevention of Meaningless Communication}
Most IoT objects are powered by battery and are therefore under strict energy constraints. It is important for these devices to preserve battery life as much as possible. As a consequence, we want to avoid sending data when no one is asking for it.\\

In order to prevent that, the Gateway subscribes to a MQTT topic that counts the number of people subscribed to the MQTT topics published by the sensor network (\$SYS/broker/subscriptions/count). When the number of subscribers change, the Gateway receives an update.\\

When no one is subscribed, the Gateway sends a message to the network, to tell the nodes that they need to stop sending data, therefore saving their battery life.

\begin{figure}
  \includegraphics[width=\linewidth]{img/communication-2.jpg}
  \caption{This schema shows how data transits through the Rime-MQTT Gateway.}
  \label{fig:communication1}
\end{figure}

\subsection{Sensor Network Configuration}
One of the requirement of this project is the possibility to dynamically modify the configuration of the sensor network. The proposed method was to by directly interacting with the Gateway via command line, but we opted for using the existing MQTT infrastructure to implement this functionality.\\

The Gateway is a subscriber for the "/config" topic, which allows it to receive configuration commands from the MQTT client that publishes configuration commands.\\

Once the MQTT message reaches the Gateway, it needs to be translated before sending it to the root node of the sensor network, after which the command will be propagated to each node.\\

Four possible commands can be sent to the sensor network :
\begin{itemize}
\item{UPDATE-MODE}
\item{PERIODIC-MODE}
\item{START-SEND}
\item{STOP-SEND}
\end{itemize}


\end{document}

